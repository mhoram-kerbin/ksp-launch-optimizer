\documentclass[11pt]{report}

\usepackage[utf8]{inputenc}
\usepackage[a4paper]{geometry}
\usepackage[pdfstartview=Fit,pdfpagelayout=SinglePage,breaklinks=true]{hyperref}
\usepackage{breakurl}
\usepackage{multicol}
\usepackage{amsfonts}
\usepackage{amsmath}
\usepackage[final]{listings}
\usepackage{makeidx}
\usepackage{textcomp}
\usepackage{ulem}

\makeindex

\lstset{ %
language=sh,                % choose the language of the code
%numbers=left,                   % where to put the line-numbers
firstnumber=auto, % automatic line numbers
columns=fullflexible,
frame=single           % adds a frame around the code
}

\newcommand{\F}[1]{\overrightarrow{F_{#1}}}
\newcommand{\Pos}{\overrightarrow{P}}
\newcommand{\Vel}{\overrightarrow{V}}
\newcommand{\absvec}[1]{\left|\left|{#1}\right|\right|}

\begin{document}

\title{Kerbal Launch Optimizer}

\maketitle

\begin{abstract}

  This software is intended to find ideal ascent trajectories for
  rockets in Kerbal Space Programm.

  Description of the optimization problem for an ideal rocket launch
  to a stable orbit in the Kerbal Space Program.

\end{abstract}

\tableofcontents

\chapter{Description of the setting}

\section{Restrictions}

In this model we assume that within a single stage all engines are
running the whole time.

\section{Declaration of names}

$t_0^p$ denotes the initial time of stage $p$.
$t_f^p$ denotes the final time of stage $p$.

\section{Constants}

\subsection{Global}

\begin{itemize}
\item $g_0$: Gravity at sealevel
\end{itemize}

\subsection{Local for initialization}

\begin{itemize}
\item $PlanetMass$: Mass of planet
\item $PlanetRadius$: Radius of planet
\item $PlanetScaleHeight$: Planets Scale Height
\item $P_0$: Pressure at sealevel
\item $SidPer$: Rotation Period
\item $SoiRadius$: Radius of sphere of influence
\item $PlanetMaxV$: Radius of sphere of influence
\item $Longitude$: Longitude of launchsite
\item $Latitude$: Latitude of launchsite
\item $Elevation$: Elevation of launchsite
\item $TargetOrbit$: Target Periapsis for orbit
\end{itemize}

\subsection{Local for each phase}

\begin{itemize}
\item $MaxThrust^p$: Maximal thrust
\item $ISP_{1Atm}^p$: ISP at $1 atm$
\item $ISP_{Vac}^p$: ISP in Vacuum
\item $PropellantMass^p$ Propellant mass
\item $TotalMass^p$ Mass at beginning of stage
\end{itemize}

\section{Functions}

\begin{itemize}
\item $V(t) := \sqrt{VelX(t)^2 + VelY(t)^2+ VelZ(t)^2}$: Orbital velocity at time $t$
\item $HorizontalV(t) := TBD$: Horizontal velocity at time $t$
\item $VerticalV(t) := TBD$: Vertical velocity at time $t$
\item $Length(x, y, z) := \sqrt{x^2 + y^2 + z^2}$: Length of a vector
\item $Versor(x, y, z) := \frac{(x, y, z)}{Length(x, y, z)}$ Versor
\item $R(t) := Length(PosX(t), PosY(t), PosZ(t))$: Distance of rocket to planet center at time $t$
\item $Periapsis(t) := \textrm{Periapsis of the orbit at time } t$
\end{itemize}

\section{States}

\begin{itemize}
\item $PosX(t)$: X-position at time $t$
\item $PosY(t)$: Y-position at time $t$
\item $PosZ(t)$: Z-position at time $t$

\item $VelX(t)$: X-velocity at time $t$
\item $VelY(t)$: Y-velocity at time $t$
\item $VelZ(t)$: Z-velocity at time $t$

\item $Mass(t)$: Mass of the rocket at time $t$
\end{itemize}

\section{Control}

\begin{itemize}
\item $ThrustX(t)$: X-thrustforce at time $t$
\item $ThrustY(t)$: Y-thrustforce at time $t$
\item $ThrustZ(t)$: Z-thrustforce at time $t$
\end{itemize}

\section{Bound Constraints}

\begin{itemize}
\item $0\leq Pos_X, Pos_Y, Pos_z \leq SoiRadius$
\item $0\leq Thrust_X, Thrust_Y, Thrust_z \leq MaxThrust^p$
\item $-PlanetMaxV\leq Vel_X, Vel_Y, Vel_z \leq PlanetMaxV^p$
\item $TotalMass^p - PropellantMass^p \leq Mass(t) \leq TotalMass^p$
\end{itemize}

\section{Path Constraints}

\begin{itemize}
\item $R(t) \geq PlanetRadius$
\item $R(t) \leq SoiRadius$
\item $Length(ThrustX(t), ThrustY(t), ThrustZ(t)) \leq MaxThrust^p$
\item In all but the final phase the velocity vector points away from the planet core???
\end{itemize}

\section{Derivative Constraints}

\begin{itemize}
\item $\dot{\overrightarrow{Position}} = \overrightarrow{Velocity}$
\item $\dot{\overrightarrow{Velocity}} = Mass^{-1}\cdot\left(\overrightarrow{Thrust} + \overrightarrow{Drag} + \overrightarrow{Gravity}\right)$
\item $\dot{Mass} = (ISP(Altitude) \cdot g_0)^{-1}\cdot\left|\overrightarrow{Thrust}\right|$
\end{itemize}
\section{Events}

\begin{itemize}
\item $ThrustX(0) \cdot PosY(0) = ThrustY(0) \cdot PosX(0)$
\item $ThrustX(0) \cdot PosZ(0) = ThrustZ(0) \cdot PosX(0)$
\item Launch position and velocity are at launchpad
\item Launch Position and Velocity Vector are 90\textdegree apart
\item Launch Mass is Total Mass of ship
\item $Periapsis(t_f^{last}) \geq TargetOrbit$
\end{itemize}

\section{Phase Linkage Constraints}

\begin{itemize}
\item $t_0^{p+1} = t_f^p$
\item $PosX(t_0^{p+1}) = PosX(t_f^p)$
\item $PosY(t_0^{p+1}) = PosY(t_f^p)$
\item $PosZ(t_0^{p+1}) = PosZ(t_f^p)$
\item $VelX(t_0^{p+1}) = VelX(t_f^p)$
\item $VelY(t_0^{p+1}) = VelY(t_f^p)$
\item $VelZ(t_0^{p+1}) = VelZ(t_f^p)$
\item $Mass(t_f^p) - (TotalMass^p - PropellantMass^p) + TotalMass^{p+1} = Mass(t_0^{p+1})$
\end{itemize}

\section{Performance Index}

\begin{itemize}
\item $J := - Mass(t_f^{final})$
\end{itemize}
or TargetPeriapsis - FinalPeriapsis, if fuel has run out.

\chapter{Implementation}

Mumps is used as linear solver.

\section{Atmospheric Pressure}

As an approximation, the pressure above atmospheric height is not 0
but calculated according to the formula for the atmosphere.

Tests showed that a separate phase is necessary for this effect.

\section{States}

The state of the simulation consists of the following items:

\begin{enumerate}
\setcounter{enumi}{-1}
\item PosX
\item PosY
\item PosZ
\item VelX
\item VelY
\item VelZ
\item Mass
\end{enumerate}

\section{Static Parameters}

\begin{enumerate}
\setcounter{enumi}{-1}
\item Planet mass
\item Planet radius
\item Planet scale height
\item Planet pressure at sealevel
\item Planet rotation period around it's own axe
\item Planet's SOI radius
\item Rocket's drag coefficient
\end{enumerate}

\section{Endpoint Cost}

\chapter{Example}

\section{Vehicle}

We use as exemplary vehicle the following linear staging rocket:

\begin{itemize}
\item Command Pod Mk 1 (840kg)
\item FL-T400 Fuel Tank (2250kg, 2000kg fuel)
\item LV 909 Liquid Fuel Engine (500kg, 50kN, 300s-390s ISP)
\item TR-18A Stack Decoupler (50kg)
\item FL-T800 Fuel Tank (4500kg, 4000kg fuel)
\item LV-T45 Liquid Fuel Engine (1500kg, 200kN, 320s-370s ISP)
\end{itemize}

This gives for the two stages:

\begin{itemize}
\item Propellant Mass: 4000kg, 2000kg
\item Total Mass: 9640kg, 3590kg
\item Thrust: 200kN, 50kN
\item $ISP_0$: 320s, 300s
\item $ISP_{Vac}$: 370s, 390s
\item Drag Coefficient 0.2, 0.2
\end{itemize}

Rocket will Launch from 0\textdegree5'49'' South, 0\textdegree0'0''
East \sout{74\textdegree33'27'' West} at an elevation of 77.1m above
sealevel.

\chapter{Notes}

\section{Experience}

\begin{itemize}
\item NODES = [10, 50] and NLP\_MAX\_ITER = 3000 seem to be a upper
  bound for 4GB memory.
\item Sometimes increasing the number of iterations results in
  oscillating control-paths without improving the endpoint cost.
\item NLP\_MAX\_ITER in the range 500 to 1000 seem to bring good
  results.
\item NODES = [10, 40] brought good results in multiple applications
\end{itemize}

\end{document}
